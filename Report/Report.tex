\documentclass{article}

\usepackage{mathtools,amssymb,amsthm}
\usepackage{minted}
\usepackage{caption}
\usepackage{subcaption}

\setlength{\abovecaptionskip}{0pt}

\title{Combinatorial Optimization Report\\[1ex]
\large Shortest Path algorithms comparison and Max Flow optimizations}
\author{Maurina Gabriele}
\date{\today}

\begin{document}

\begin{titlepage}
\maketitle
\end{titlepage}

\begin{abstract}
This document contains a report on the project created by Gabriele Maurina for the exam Combinatorial Optimization.
The report is divided in two parts. The first is about the Shortest Path Problem. It contains an explanation of the problem,
an explanation of the algorithms considered in the project and their implementations, and an evaluation of their performance
with comparison between them. The algorithms implemented and compared in the first part are Dijkstra, Bellman-Ford and Moore.
The results obtained show that Moore's algorithm is consistently the fastest, followed by a closed second Dijkstra's algorithm
and Bellman-Ford's algorithm is consistently far behind.
The second part is about the Max Flow problem. It contains an explanation of the problem, an explanation of the algorithms considered in the project and
their implementations, and an evaluation of their performance with comparison between them. Particularly, this part of the project tried to
use bidirectional extension of labels to improve the Ford-Fulkerson algorithm in its Breadth First Search (BFS) and Dijkstra variations. Overall the algorithms implemented
and compared in the second part are Ford-Fulkerson with BFS for Shortest Augmenting Path, Ford-Fulkerson with bidirectional BFS for Shortest Augmenting Path,
Ford-Fulkerson with Dijkstra for Maximum Capacity Augmenting Path and Ford-Fulkerson with bidirectional Dijkstra for Maximum Capacity Augmenting Path.
The results obtained show that the bidirectional Dijkstra variation is consistently the fastest, with its monodirectional counterpart following closely.
The BFS variation is consistently the slowest, whereas it bidirectional counterpart shows mixed results, being as fast as Dijkstra's variation
when the average path length is short, i.e. with less than ten edges between source and destination, and being as slow as its monodirectional
counterpart when the average path length is long, i.e. with hundreds of edges between source and destination.
\end{abstract}

\tableofcontents

\input ShortestPath
\input MaxFlow
\input FiguresAndTables

\end{document}
