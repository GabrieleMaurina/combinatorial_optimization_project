\section{Shortest Path}
\subsection{Problem}
Given a directed graph $G=\{V,E\}$, where $V$ is the set of vertices, $E$ is the set of edges, $n=||V|$ and $m=|E|$.
Let $s \in V$ be the source and let $c: E \to \mathbb{R}$ be a cost function.
\vspace{1em}\\
\textbf{Shortest Path Problem:} find all minimum cost paths from $s$ to all vertices in $V$.

\subsection{Implementation}
There exist several algorithms to solve this problem. The algorithms implemented and tested are Dijkstra's algorithm, Bellman-Ford's algorithm and Moore's algorithm.
The programming language used is C++ because it offers great control over it's data structures and has a rich standard library.

\subsection{Graph representation}
In order to execute any algorithm for the Shortest Path Problem it is necessary to have a data structure to represent a graph.
For simplicity of implementation and to save space, an out-star representation is used.
Three structures are used: Graph, Vertex and Edge. Where Graph contains an array of Vertex and an array of Edge, each Vertex contains an array of the out going Edge
and each Edge contains a reference to the Vertex it comes from, a reference to the Vertex it goes to and its cost. For simplicity the cost is stored as an 8 bit
unsigned integer, i.e. only positive costs are allowed. Both Vertex and Edge contain a field id,
which is their position in the array in the graph. This representation saves space when the graph is sparse, compared to an adjacency matrix.

The structures used are:
\begin{minted}{c++}
struct Vertex{
	ui id;
	vector<Edge*> edges;
};

struct Edge{
	ui id;
	sui cost;
	Vertex* from;
	Vertex* to;
	Edge(ui _id,sui _cost,Vertex* _from,Vertex* _to);
};

struct Graph{
	vector<Vertex> vertices;
	vector<Edge> edges;
	Graph(ui N,f density,ui delta=0);
};
\end{minted}
Where $s$ is always the first Vertex.

The graph is generated randomly given size, density and delta, according to the following pseudocode:
\begin{minted}{python}
create array of vertices of size size;
for each vertex v1:
    for each vertex v2:
        if delta = 0 or abs(v1.id-v2.id) <= delta:
            if rand_float(0,1) < density:
                create an edge with random cost between v1 and v2;
\end{minted}
Delta represents the maximum difference in id between vertices. When delta is zero it is discarded. Delta is necessary if we want to create a
certain kind of graph, i.e. one whose shortest path between first and last vertex has many edges. Density is the density of the graph, only when delta is zero.

\subsection{Dijkstra's algorithm}
Dijkstra's algorithm is implemented in its simplest form with an array, (no priority queue). Hence its execution time is $O(n^2)$.
An array $cost$ is used to store the distance from $s$ to every vertex.
An array $pred$ is used to store the predecessor of each vertex in a path with minimum cost from $s$ to itself.
An array $flag$ is used to store whether a path of minimum cost to a vertex has been found.

The implementation is:
\begin{minted}{c++}
void dijkstra(const Graph& g,const ui s){
	const ui N = g.vertices.size();
	vector<ui> cost(N,INF);
	vector<ui> pred(N,s);
	vector<bool> flag(N,false);
	cost[s] = 0;
	for(ui i=0;i<N;i++){
		ui cmin = INF;
		ui idmin = 0;
		for(ui j=0;j<N;j++)
		if(!flag[j]&&cost[j]<cmin){
			idmin = j;
			cmin = cost[j];
		}
		flag[idmin] = true;
		for(auto e:g.vertices[idmin].edges)
		if(!flag[e->to->id])
		if(cost[idmin]+e->cost<cost[e->to->id]){
			cost[e->to->id] = cost[idmin]+e->cost;
			pred[e->to->id] = idmin;
		}
	}
}
\end{minted}

\subsection{Bellman-Ford's algorithm}
Bellman-Ford's algorithm loops $|V|$ times over the entire Edge array. Hence its execution time is $O(nm)$.
An array $cost$ is used to store the distance from $s$ to every vertex.
An array $pred$ is used to store the predecessor of each vertex in a path with minimum cost from $s$ to itself.

The implementation is:
\begin{minted}{c++}
void bellman_ford(const Graph& g,const ui s){
	const ui N = g.vertices.size();
	vector<ui> cost(N,INF);
	vector<ui> pred(N,s);
	cost[s] = 0;
	for(ui i=1;i<N;i++)
	for(auto e:g.edges)
	if(cost[e.from->id]+e.cost<cost[e.to->id]){
		cost[e.to->id] = cost[e.from->id]+e.cost;
		pred[e.to->id] = e.from->id;
	}
}
\end{minted}

\subsection{Moore's algorithm}
Moore's algorithm is implemented with a FIFO queue. Its execution time is $O(nm)$.
The FIFO queue is implemented combinig a queue and a set for $O(1)$ lookup. Its structure is:
\begin{minted}{c++}
struct MooreFifoQueue{
	queue<ui> q;
	unordered_set<ui> us;
	void push(const ui value);
	ui pop();
	bool empty();
	bool contains(const ui value);
};
\end{minted}

An array $cost$ is used to store the distance from $s$ to every vertex.
An array $pred$ is used to store the predecessor of each vertex in a path with minimum cost from $s$ to itself.

The implementation of Moore's algorithm is:
\begin{minted}{c++}
void moore(const Graph& g,const ui s){
	const ui N = g.vertices.size();
	vector<ui> cost(N,INF);
	vector<ui> pred(N,s);
	cost[s] = 0;
	MooreFifoQueue q;
	q.push(s);
	while(!q.empty()){
		ui i = q.pop();
		for(auto e:g.vertices[i].edges)
		if(cost[i]+e->cost<cost[e->to->id]){
			cost[e->to->id] = cost[i]+e->cost;
			pred[e->to->id] = i;
			if(!q.contains(e->to->id))
			q.push(e->to->id);
		}
	}
}
\end{minted}

\subsection{Results}
The three previously discussed algorithms are evaluated in order to test their efficiency w.r.t execution time. All experiments are conducted on
a laptop running Fedora Workstation 32 with an Intel® Core™ i7-2670QM CPU @ 2.20GHz×8 and 6 GB of ram.
The evaluation is performed on large graphs with up to 4thousands vertices and 1.6 million edges. Different levels of density and delta are used.
Table \ref{table:spp} summarizes the results obtained.

Figures \ref{fig:spp1},\ref{fig:spp2} and \ref{fig:spp3} show the execution time of the algorithms
with graphs of delta=0 and increasing size and density.
Moore's algorithm is fastest, followed by Dijkstra's algorithm. Bellman-Ford's algorithm is by far the slowest.

Figures \ref{fig:spp4},\ref{fig:spp5} and \ref{fig:spp6} show the execution time of the algorithms
with graphs of density=0.8 and increasing size and delta.
Moore's algorithm is fastest, followed by Dijkstra's algorithm. Bellman-Ford's algorithm is by far the slowest.

The results obtained are consistent across all kinds of graphs tested.
