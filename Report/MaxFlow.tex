\section{Max Flow}
\subsection{Problem}
Given a directed graph $G=\{V,E\}$, where $V$ is the set of vertices, $E$ is the set of edges, $n=||V|$ and $m=|E|$.
Let $s \in V$ be the source, let $t \in V$ be the sink and let $u: E \to \mathbb{R}_+$ be a capacity function.
Let $z$ be the flow, i.e. a quantity that can traverse each edge from a vertex to another starting in $s$ and finishing in $t$ such that it never exceeds the capacity of an edge,
it is not negative and the flow entering a vertex is equal to the flow leaving it, unless the vertex is $s$ or $t$. Furthermore the flow leaving $s$ is equal to the flow entering $t$.
The non-negative continuous variable $x_{ij}$ indicates the flow on the edge $(i,j)\in E$.
\vspace{1em}\\
\textbf{Max Flow Problem:} maximize $z$ such that:
\[
\sum_{i\in V:(i,j)\in E} x_{ij} - \sum_{i\in V:(j,i)\in E} x_{ji} =
\begin{cases}
    z & \text{if } i=s\\
    0 & \text{if } i\neq s,t\\
    -z & \text{if } i=t
\end{cases}
\]

\[0 \leq x_{ij} \leq u_{ij} \;\;\;\;\;\;\; \forall (i,j)\in E \]

\subsection{Implementation}
Ford-Fulkerson's algorithm solves the Max Flow problem. It's basic version uses BFS to find the shortest augmenting path. Its overall execution time is $O(nm^2)$.
The project implements Ford-Fulkerson's algorithm in 4 variations: with BFS, with bidirectional BFS, with Dijkstra's algorithm to find the maximum capacity augmenting paths
and with bidirectional Dijkstra's algorithm again to find the maximum capacity augmenting paths.
The programming language used is again C++ because it offers great control over it's data structures and has a rich standard library.

\subsection{Graph representation}
The graph representation is the same as with the Shortest Path Problem with the exception that now each Vertex holds
an array of outgoing edges as well as incoming ones. This is necessary because the algorithms used need to traverse said edges in both directions.

The newly Vertex structure is:
\begin{minted}{c++}
struct Vertex{
	ui id;
	vector<Edge*> out_edges;
	vector<Edge*> in_edges;
};
\end{minted}

The graph generation is the same as with the Shortest Path Problem with the exception that now when
creating an Edge it is also necessary to store it in the receiving Vertex.

\subsection{Ford-Fulkerson's algorithm}
Ford-Fulkerson algorithm is implemented in an abstract structure so that its variant can extend it and implement the virtual method
to compute the augmenting path in the manner that they so choose. This way we can separate Ford-Fulkerson's algorithm from the other algorithms
that find the augmenting path.

The implementation is:
\begin{minted}{c++}
void FordFulkerson::run(){
	while(compute_path()){
		sui delta = 255;
		for(ui i=0; i<path.size();i++)
			if(path_dir[i] && path[i]->u-x[path[i]->id]<delta)
				delta = path[i]->u-x[path[i]->id];
			else if(!path_dir[i] && x[path[i]->id]<delta)
				delta = x[path[i]->id];
		for(ui i=0; i<path.size();i++)
			if(path_dir[i])
				x[path[i]->id] += delta;
			else
				x[path[i]->id] -= delta;
	}
}
\end{minted}

\subsection{Breadth First Search algorithm}
BFS algorithm is used to find the shortest augmenting path in the graph. It's complexity is $O(m)$, since it visits all edges at most once.
An array $visited$ is used to store whether a vertex has already been visited.
An array $pred$ is used to store the predecessor of each vertex in a path from $s$ to $t$.
An array $dir$ is used to store the direction of the path in the edge preceding each vertex.

The implementation is:
\begin{minted}{c++}
bool Bfs::compute_path(){
	path.clear();
	path_dir.clear();
	queue<ui> q;
	vector<bool> visited(g->vertices.size(),false);
	vector<ui> pred(g->vertices.size());
	vector<bool> dir(g->vertices.size());
	q.push(s);
	visited[s] = true;
	while(!q.empty()){
		ui i = q.front();
		q.pop();
		if(i==t){
			while(i!=s){
				path.push_back(&g->edges[pred[i]]);
				path_dir.push_back(dir[i]);
				if(dir[i]) i = g->edges[pred[i]].from->id;
				else i = g->edges[pred[i]].to->id;
			}
			return true;
		}
		for(auto e:g->vertices[i].out_edges)
			if(!visited[e->to->id])
			if(e->u-x[e->id]>0){
				visited[e->to->id] = true;
				q.push(e->to->id);
				pred[e->to->id] = e->id;
				dir[e->to->id] = true;

			}
		for(auto e:g->vertices[i].in_edges)
			if(!visited[e->from->id])
			if(x[e->id]>0){
				visited[e->from->id] = true;
				q.push(e->from->id);
				pred[e->from->id] = e->id;
				dir[e->from->id] = false;
			}
	}
	return false;
}
\end{minted}

\subsection{Bidirectional Breadth First Search algorithm}
Bidirectional BFS algorithm is used to find the shortest augmenting path in the graph. It's complexity is $O(m)$, since it visits all edges at most once.
At its core it is composed of two BFS algorithms that run in parallel, one starting from $s$ and one starting from $t$,
and stopping only when they both have visited the same vertex. It generally runs faster than a simple BFS, but it requires double the space, indeed every
data structure needs to be allocated twice.
Two arrays $vs$ and $vt$ are used to store whether a vertex has already been visited respectively by the BFS starting in $s$ and the BFS starting in $t$.
Two arrays $ps$ and $pt$ are used to store the predecessor of each vertex in a path from $s$ to itself
respectively by the BFS starting in $s$ and the BFS starting in $t$.
Two arrays $ds$ and $dt$ are used to store the direction of the path in the edge preceding each vertex
respectively by the BFS starting in $s$ and the BFS starting in $t$.

The implementation is:
\begin{minted}{c++}
bool BfsBi::compute_path(){
	path.clear();
	path_dir.clear();
	queue<ui> qs;
	queue<ui> qt;
	vector<bool> vs(g->vertices.size(),false);
	vector<bool> vt(g->vertices.size(),false);
	vector<ui> ps(g->vertices.size());
	vector<ui> pt(g->vertices.size());
	vector<bool> ds(g->vertices.size());
	vector<bool> dt(g->vertices.size());
	qs.push(s);
	qt.push(t);
	vs[s] = true;
	vt[t] = true;
	while(!qs.empty()&&!qt.empty()){
		ui is = qs.front();
		ui it = qt.front();
		qs.pop();
		qt.pop();
		if(vt[is]){
			reconstruct_path(is,ps,pt,ds,dt);
			return true;
		}
		if(vs[it]){
			reconstruct_path(it,ps,pt,ds,dt);
			return true;
		}
		for(auto e:g->vertices[is].out_edges)
			if(!vs[e->to->id])
			if(e->u-x[e->id]>0){
				vs[e->to->id] = true;
				qs.push(e->to->id);
				ps[e->to->id] = e->id;
				ds[e->to->id] = true;

			}
		for(auto e:g->vertices[is].in_edges)
			if(!vs[e->from->id])
			if(x[e->id]>0){
				vs[e->from->id] = true;
				qs.push(e->from->id);
				ps[e->from->id] = e->id;
				ds[e->from->id] = false;
			}
		for(auto e:g->vertices[it].out_edges)
			if(!vt[e->to->id])
			if(x[e->id]>0){
				vt[e->to->id] = true;
				qt.push(e->to->id);
				pt[e->to->id] = e->id;
				dt[e->to->id] = false;
			}
		for(auto e:g->vertices[it].in_edges)
			if(!vt[e->from->id])
			if(e->u-x[e->id]>0){
				vt[e->from->id] = true;
				qt.push(e->from->id);
				pt[e->from->id] = e->id;
				dt[e->from->id] = true;
			}
	}
	return false;
}
\end{minted}

\subsection{Dijkstra's algorithm}
A variation of Dijkstra's algorithm for the shortest path is used in this project to compute the maximum capacity augmenting path.
Dijkstra's algorithm in its optimal form using a priority queue runs in $O(m\log n)$. The implementation used in this project does use a priority queue,
however it does not decrease the key of vertices already inside said queue, instead it simply adds another entry. This simplifies the implementation of the queue.
An array $c$ is used to store maximum residual capacity going through each vertex.
An array $pred$ is used to store the predecessor of each vertex in a path from $s$ to $t$.
An array $dir$ is used to store the direction of the path in the edge preceding each vertex.

The implementation is:
\begin{minted}{c++}
bool Dijkstra::compute_path(){
	path.clear();
	path_dir.clear();
	priority_queue<pair<sui,ui>> q;
	vector<sui> c(g->vertices.size(),0);
	vector<ui> pred(g->vertices.size());
	vector<bool> dir(g->vertices.size());
	q.push({MAX_CAPACITY,s});
	c[s] = MAX_CAPACITY;
	while(!q.empty()){
		sui ci = q.top().first;
		ui i = q.top().second;
		q.pop();
		if(ci<c[i]) continue;
		if(i==t){
			while(i!=s){
				path.push_back(&g->edges[pred[i]]);
				path_dir.push_back(dir[i]);
				if(dir[i]) i = g->edges[pred[i]].from->id;
				else i = g->edges[pred[i]].to->id;
			}
			return true;
		}
		for(auto e:g->vertices[i].out_edges){
			sui ec = e->u-x[e->id];
			sui nc = min(ec,c[i]);
			if(nc>c[e->to->id]){
				c[e->to->id]=nc;
				q.push({nc,e->to->id});
				pred[e->to->id] = e->id;
				dir[e->to->id] = true;
			}
		}
		for(auto e:g->vertices[i].in_edges){
			sui ec = x[e->id];
			sui nc = min(ec,c[i]);
			if(nc>c[e->from->id]){
				c[e->from->id]=nc;
				q.push({nc,e->from->id});
				pred[e->from->id] = e->id;
				dir[e->from->id] = false;
			}
		}
	}
	return false;
}
\end{minted}

\subsection{Bidirectional Dijkstra's algorithm}
A bidirectional Dijkstra's algorithm is used in this project to compute the maximum capacity augmenting path. Its complexity is the same as normal Dijkstra.
Indeed at its core it is composed of two Dijkstra's algorithms that run in parallel, one starting from $s$ and one starting from $t$,
and stopping only when they both have visited the same vertex. It generally runs faster than a simple Dijkstra's algorithm, but it requires double the space, indeed every
data structure needs to be allocated twice.
Two arrays $vs$ and $vt$ are used to store whether a vertex has already been visited respectively by the algorithm starting in $s$ and the algorithm starting in $t$.
Two arrays $cs$ and $ct$ are used to store maximum residual capacity going through each vertex respectively by the algorithm starting in $s$ and the algorithm starting in $t$.
Two arrays $ps$ and $pt$ are used to store the predecessor of each vertex in a path from $s$ to $t$ respectively by the algorithm starting in $s$ and the algorithm starting in $t$.
Two arrays $ds$ and $dt$ are used to store the direction of the path in the edge preceding each vertex respectively by the algorithm starting in $s$ and the algorithm starting in $t$.

The implementation is:
\begin{minted}{c++}
bool DijkstraBi::compute_path(){
	path.clear();
	path_dir.clear();
	priority_queue<pair<sui,ui>> qs;
	priority_queue<pair<sui,ui>> qt;
	vector<sui> cs(g->vertices.size(),0);
	vector<sui> ct(g->vertices.size(),0);
	vector<bool> vs(g->vertices.size(),false);
	vector<bool> vt(g->vertices.size(),false);
	vector<ui> ps(g->vertices.size());
	vector<ui> pt(g->vertices.size());
	vector<bool> ds(g->vertices.size());
	vector<bool> dt(g->vertices.size());
	qs.push({MAX_CAPACITY,s});
	qt.push({MAX_CAPACITY,t});
	cs[s] = MAX_CAPACITY;
	ct[t] = MAX_CAPACITY;
	while(!qs.empty()&&!qt.empty()){
		sui cis = qs.top().first;
		ui is = qs.top().second;
		qs.pop();
		sui cit = qt.top().first;
		ui it = qt.top().second;
		qt.pop();
		vs[is] = true;
		vt[it] = true;

		if(vt[is]){
			reconstruct_path(is,ps,pt,ds,dt);
			return true;
		}
		if(vs[it]){
			reconstruct_path(it,ps,pt,ds,dt);
			return true;
		}

		if(cis=cs[is]){
			for(auto e:g->vertices[is].out_edges){
				sui ec = e->u-x[e->id];
				sui nc = min(ec,cs[is]);
				if(nc>cs[e->to->id]){
					cs[e->to->id]=nc;
					qs.push({nc,e->to->id});
					ps[e->to->id] = e->id;
					ds[e->to->id] = true;
				}
			}
			for(auto e:g->vertices[is].in_edges){
				sui ec = x[e->id];
				sui nc = min(ec,cs[is]);
				if(nc>cs[e->from->id]){
					cs[e->from->id]=nc;
					qs.push({nc,e->from->id});
					ps[e->from->id] = e->id;
					ds[e->from->id] = false;
				}
			}
		}
		if(cit=ct[it]){
			for(auto e:g->vertices[it].out_edges){
				sui ec = x[e->id];
				sui nc = min(ec,ct[it]);
				if(nc>ct[e->to->id]){
					ct[e->to->id]=nc;
					qt.push({nc,e->to->id});
					pt[e->to->id] = e->id;
					dt[e->to->id] = false;
				}
			}
			for(auto e:g->vertices[it].in_edges){
				sui ec = e->u-x[e->id];
				sui nc = min(ec,ct[it]);
				if(nc>ct[e->from->id]){
					ct[e->from->id]=nc;
					qt.push({nc,e->from->id});
					pt[e->from->id] = e->id;
					dt[e->from->id] = true;
				}
			}
		}
	}
	return false;
}
\end{minted}

\subsection{Results}
The previously discussed algorithms are evaluated in order to test their efficiency w.r.t execution time. All experiments are conducted on
a laptop running Fedora Workstation 32 with an Intel® Core™ i7-2670QM CPU @ 2.20GHz×8 and 6 GB of ram.
The evaluation is performed on large graphs with up to 4thousands vertices and 1.6 million edges. Different levels of density and delta are used.
Table \ref{table:mf} summarizes the results obtained.

Figures \ref{fig:mf1},\ref{fig:mf2} and \ref{fig:mf3} show the execution time of the algorithms
with graphs of delta=0 and increasing size and density.
The bidirectional algorithms are fastest, followed by Dijkstra's algorithm. Monodirectional BFS is by far the slowest.

Figures \ref{fig:mf4},\ref{fig:mf5} and \ref{fig:mf6} show the execution time of the algorithms
with graphs of density=0.8 and increasing size and delta. This variant of graph has on average a longer shortest path from $s$ to $t$.
Both Dijkstra's algorithms are consistently fastest and conversely both BFS algorithms are consistently slowest.
In both cases the bidirectional version seems to have a slight edge over its monodirectional counterpart.
